\documentclass[12pt]{report}
\usepackage[utf8]{vietnam}
\usepackage{amsmath}
\usepackage{verbatim}
\usepackage{mathtools}
\DeclarePairedDelimiter{\ceil}{\lceil}{\rceil}

\begin{document}

\begin{align*}
    &P(x) = a_0 + a_1 x + a_2 x^2 + \dots + a_{m - 1} x^{m - 1} = 
    \sum_{i = 0}^{m - 1} a_i x ^ i \\
    &Q(x) = b_0 + b_1 x + b_2 x^2 + \dots + b_{n - 1} x^{n - 1} = 
    \sum_{j = 0}^{n - 1} b_j x^j \\
    &P(x) \times Q(x) = c_0 + c_1 x + c_2 x^2 + \dots 
    + c_{p - 1} x^{p - 1} =
    \sum_{k = 0}^{p - 1} c_k x^k 
\end{align*}

\begin{align*}
    &\Rightarrow c_k = \sum_{i = 0} ^ {k} a_i \times b_{k - i} \\
    &\textrm{Với: } \quad i \le m - 1 \\
    &\textrm{Và: } \quad k - i \le n - 1 \Leftrightarrow i \ge k - n + 1\\
    &\textrm{Hay: } \quad 
    c_k = \sum_{i = \max(0, k - n + 1)} ^ {\min(k, m - 1)} 
    a_i \times b_{k - i} 
    \quad\quad\forall k = \overline{0, m + n - 2} \\
    &\textrm{Thời gian tính toán: } \mathcal{O}\left(m \times n\right)
\end{align*}

\begin{align*}
    &\textrm{Đặt: } \\
    &\quad\quad p = m + n - 1 \\
    &\quad\quad N = \ceil[\Big]{\log_2 (p)} \\
    &\quad\quad W_u = e ^ {-j\frac{2 \pi u}{N}}
\end{align*}

\begin{align*}
    &\textrm{Ta có: } \\
    &\quad\quad P(W_u) = \sum_{i = 0}^{m - 1} a_i W_u^i \\
    &\quad\quad Q(W_u) = \sum_{j = 0}^{n - 1} b_j W_u^j \\
    &\quad\quad P(W_u) \times Q(W_u) = \sum_{k = 0}^{p - 1} c_k W_u^k
\end{align*}

\begin{align*}
    &\textrm{Nếu đặt: } \\
    &\quad\quad a_i = 0 \quad \forall i = \overline{m, N - 1} \\
    &\quad\quad b_j = 0 \quad \forall j = \overline{n, N - 1} \\
    &\quad\quad c_k = 0 \quad \forall k = \overline{p, N - 1} \\
\end{align*}

\begin{align*}
    &\textrm{Thì ta có: } \\
    &\quad\quad c_k = \sum_{i = 0} ^ {k} a_i \times b_{k - i}
    \quad\quad \forall k = \overline{0, N - 1}
\end{align*}

\begin{align*}
    &\textrm{Và: } \\
    &\quad\quad P(W_u) = \sum_{i = 0}^{N - 1} a_i W_u^i 
    = \mathcal{F}[a_i]_u \\
    &\quad\quad Q(W_u) = \sum_{i = 0}^{N - 1} b_i W_u^i 
    = \mathcal{F}[b_i]_u \\
    &\quad\quad P(W_u) \times Q(W_u) = \sum_{i = 0}^{N - 1} c_i W_u^i
    = \mathcal{F}[c_i]_u
\end{align*}

\begin{align*}
    \Rightarrow c_i 
    &= \mathcal{F}^{-1} \left[ \mathcal{F}[c_i]_u \right]_i \\
    &= \mathcal{F}^{-1} \left[ P(W_u) \times Q(W_u) \right]_i \\
    &= \mathcal{F}^{-1} \left[ 
        \mathcal{F}[a_i]_u \times
        \mathcal{F}[b_i]_u \right]_i
\end{align*}

\begin{align*}
    &\textrm{Thời gian tính toán: } \\
    &\quad\quad \mathcal{F}[a_i]_u \quad\quad \mathcal{O}(N \log N) \\
    &\quad\quad \mathcal{F}[b_i]_u \quad\quad \mathcal{O}(N \log N) \\
    &\quad\quad P(W_u) \times Q(W_u) \quad\quad \mathcal{O}(N) \\
    &\quad\quad \mathcal{F}^{-1}[P(W_u) \times Q(W_u)]_i 
    \quad\quad \mathcal{O}(N \log N) \\
    &\textrm{Tổng: } \quad \mathcal{O}(N \log N) < 
                \mathcal{O}\left(m \times n\right)
\end{align*}

\begin{align*}
    &\textrm{Số các segment từ vị trí thứ j của số 1, ứng với k 
    số 1 là: } \\
    &\quad\quad C_{j - 1} \times C_{j - 1 + k} 
\end{align*}

Gọi $C_j$ là số các phần tử $0$ sau phần tử $1$ thứ $j$, cộng $1$. 
($j = \overline{1, m})$ và $C_0$ là phần tử $0$ bắt đầu xâu M, cộng $1$.

\begin{align*}
    &\Rightarrow \textrm{Tổng số segment ứng với mỗi k: } \\
    &\quad\quad X_k = \sum_{j = 1}^{m + 1 - k} C_{j - 1} \times C_{j - 1 + k} 
\end{align*}\\[1cm]

% FFT
\noindent Với:  $N = 2^p$
\begin{align*}
    X_u &= \mathcal{F}[x_i]_u = \sum_{i = 0}^{N - 1} x_i 
    e^{-j\frac{2\pi i u}{N}} \\
    &= \sum_{i = 0}^{\frac{N}{2} - 1} x_{2i} e^{-j\frac{2\pi 2 i u}{N}}
    + \sum_{i = 0}^{\frac{N}{2} - 1} x_{2i + 1} 
    e^{-j\frac{2\pi (2i + 1) u}{N}} \\
    &= \sum_{i = 0}^{\frac{N}{2} - 1} x_{2i} e^{-j\frac{2\pi i u}{N/2}}
    + e^{-j\frac{2\pi u}{N}} 
    \sum_{i = 0}^{\frac{N}{2} - 1} x_{2i+ 1} 
    e^{-j\frac{2\pi i u}{N / 2}}
\end{align*}\\[1cm]

\noindent Đặt: $a_i$ là dãy các số có chỉ số chẵn lấy ra từ $x_i$. \\
\hspace*{22pt} $b_i$ là dãy các số có chỉ số lẻ lấy ra từ $x_i$.

\noindent Ta có: 
\begin{align*}
    \sum_{i = 0}^{\frac{N}{2} - 1} x_{2i} e^{-j\frac{2\pi i u}{N/2}}
    &= \sum_{i = 0}^{\frac{N}{2} - 1} a_i e^{-j\frac{2\pi i u}{N/2}} \\
    &= \mathcal{F}[a_i]_u 
    \quad\quad\left(\textrm{Có }\frac{N}{2}\textrm{ phần tử}\right) \\
    \sum_{i = 0}^{\frac{N}{2} - 1} x_{2i + 1} e^{-j\frac{2\pi i u}{N/2}}
    &= \sum_{i = 0}^{\frac{N}{2} - 1} b_i e^{-j\frac{2\pi i u}{N/2}} \\
    &= \mathcal{F}[b_i]_u
    \quad\quad\left(\textrm{Có }\frac{N}{2}\textrm{ phần tử}\right)
\end{align*}
Nhưng lại có: $u = \overline{0, N - 1}$ \\[1cm]

\newpage
\noindent Tuy nhiên, ta có thể thấy:
\begin{align*}
    e^{-j\frac{2\pi i u}{N/2}} 
    &= e^{-j\frac{2\pi i u}{N/2} + j 2\pi i}  \\
    &= e^{-j\frac{2\pi i (u - N/2)}{N/2}}
\end{align*}
Tuần hoàn với chu kì $\frac{N}{2}$ \\
Nên ta chỉnh cần tính $\frac{N}{2}$ giá trị $u$ cho 
$\mathcal{F}[a_i]_u$ và $\mathcal{F}[b_i]_u$ là đủ. \\[1cm]

\noindent Từ đó: \\
Nếu xét với $u = \overline{0, \frac{N}{2} - 1}$
\begin{align*}
    \mathcal{F}[x_i]_u &= \mathcal{F}[a_i]_u 
    + e^{-j\frac{2\pi u}{N}} \mathcal{F}[b_i]_u \\
    \mathcal{F}[x_i]_{u + N/2} &= \mathcal{F}[a_i]_u 
    + e^{-j\frac{2\pi (u + N/2)}{N}} \mathcal{F}[b_i]_u \\
    &= \mathcal{F}[a_i]_u 
    - e^{-j\frac{2\pi u}{N}} \mathcal{F}[b_i]_u
\end{align*}\\[1cm]

\noindent Thời gian tính toán:  \\
Gọi $T(n)$ là thời gian tính cho giải thuật này.
\begin{align*}
    &T\left(n\right) = 2 T\left(\frac{n}{2}\right) 
    + \Theta \left(n\right) \\
    &\Rightarrow T(n) = \Theta \left( n \log n\right)
\end{align*}

% Example
\begin{align*}
    W_{s, u} = e^{-j\frac{2\pi u}{2 (1 \ll s)}}
\end{align*}

\begin{align*}
    \mathcal{F}[x_i]_u &= \mathcal{F}[a_i]_u 
    + e^{-j\frac{2\pi u}{N}} \mathcal{F}[b_i]_u \\
    \mathcal{F}[x_i]_{u + N/2} &= \mathcal{F}[a_i]_u 
    - e^{-j\frac{2\pi u}{N}} \mathcal{F}[b_i]_u
\end{align*}\\[1cm]


\end{document}
