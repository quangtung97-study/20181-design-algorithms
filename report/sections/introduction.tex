\documentclass[../report.tex]{subfiles}

\begin{document}
Biến đổi Fourier (Fourier Transform) là một trong những công 
cụ toán học quan trọng bậc nhất với khoa học nói chung 
và xử lý tín hiệu nói riêng. 

Trong toán học có 4 loại phép biến đổi Fourier khác nhau: 
\begin{itemize}
    \item Biến đổi Fourier (liên tục): Là loại biến đổi 
        với tín hiệu liên tục, không tuần hoàn. 
    \item Biến đổi Fourier thời gian rời rạc (Discete-time 
            Fourier Transform: Tín hiệu là rời rạc, không tuần hoàn, 
            phổ tuần hoàn. 
    \item Chuỗi Fourier (Fourier Series):
        Tín hiệu là liên tục, tuần hoàn, phổ rời rạc. 
    \item Biến đổi Fourier rời rạc (Discrete Fourier Transform): 
        Tín hiệu là rời rạc, tuần hoàn, phổ rời rạc, tuần hoàn. 
\end{itemize}

Trong đó chỉ có phép biến đổi Fourier rời rạc là thực hiện 
được trên máy tính vì đầu vào và đầu ra đều là rời rạc. Một trong
những thuật toán 
nhanh nhất để thực hiện phép biến đổi này có tên gọi là thuật toán 
biến đổi Fourier nhanh (Fast Fourier Transform). Thuật toán FFT 
ngoài việc được sử dụng rộng dãi trong xử lý tín hiệu số thì còn 
có thể được sử dụng trong các bài toán nhân nhanh đa thức được 
trình bày dưới đây. 

Không những vậy, nghiên cứu chỉ ra rằng mắt và tai người, 
động vật có "cài đặt" sẵn thuật toán biến đổi 
Fourier để giúp chúng ta nhìn và nghe, 
vì vậy nó được GS Ronald Coifman của đại học Yale 
gọi là Phương pháp phân tích dữ liệu của 
tự nhiên ("Nature's way of analyzing data") 
\cite{vnoi-fft} \cite{fourier-transform}
\end{document}
